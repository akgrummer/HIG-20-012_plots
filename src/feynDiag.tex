%%%%%%%%%%%%%%%%%%%%%%%%%%%%%%%%%%%%%%%
% author: Aidan Grummer
% adapted from examples found in:
% https://osksn2.hep.sci.osaka-u.ac.jp/~taku/osx/feynmp/fmfsamples.pdf
% https://ctan.mirrors.hoobly.com/macros/latex/contrib/feynmf/fmfman.pdf
% https://www.overleaf.com/learn/latex/Feynman_diagrams#An_overview_of_using_the_feynmf-based_packages
%
% compiled in overleaf
% using PDFlatex compiler and TEX Live version 2024
%%%%%%%%%%%%%%%%%%%%%%%%%%%%%%%%%%%%%%%
\documentclass[margin=10mm]{standalone}
% \usepackage[compat=1.0.0]{tikz-feynman}
\usepackage{feynmp-auto}
\begin{document}

% it was helpful to draw the diagram on paper before starting
% goal was to label leftmost verices with i (incoming) and
% rightmost vertices with o (outgoing)
% internal vertices alphabetically: a, b, c and so on
\begin{fmffile}{simple_box}
% must have the start in fmfgraph* to render the labels
\begin{fmfgraph*}(240,110)
    \fmfstraight
    % define left (incoming) 'external' vertices
    \fmfleft{i1,i2}
    % define right (outgoing) 'external' vertices
    \fmfright{o1,o2,o3,o4}
    % define top and bottom vertices
    \fmftop{p2}
    \fmfbottom{p1}
    % set positions of the external vertices
    \fmfforce{(0,0.15h)}{i1}
    \fmfforce{(0,0.85h)}{i2}
    \fmfforce{(0.4w,0)}{p1}
    \fmfforce{(0.4w,h)}{p2}
    % set labels and label positions of external vertices
    \fmfv{label.dist=5,label.angle=180,label=$p$}{i1}
    \fmfv{label.dist=5,label.angle=180,label=$p$}{i2}
    \fmfv{label.dist=5,label.angle=0,label=$\bar{b}$}{o1}
    \fmfv{label.dist=5,label.angle=0,label=$b$}{o2}
    \fmfv{label.dist=5,label.angle=0,label=$\bar{b}$}{o3}
    \fmfv{label.dist=5,label.angle=0,label=$b$}{o4}


    % protons (center lines)
    \fmf{fermion,tension=2}{i1,a1}
    \fmf{fermion,tension=2}{i2,a2}
    % invisible line to help with positioning
    \fmf{phantom,tension=-0.3}{a1,a2}
    % add gluons with labels
    \fmf{gluon,label.dist=10,label=$g$}{a1,b1}
    \fmf{gluon,label.dist=10,label=$g$}{a2,b1}
    % label.angle did not work for particle line labels
    % \fmf{gluon,label.dist=10,label=$g$,label.angle=0}{a2,b1}

    % outgoing proton, added extra vertex to help with massage alignment of
    % the three proton lines (see code at end)
    % note labels can have quote mark <'> in them
    \fmf{plain,tension=10}{a2,a2'}
    \fmf{plain,tension=10}{a1,a1'}
    \fmf{plain,tension=1}{a2',p2}
    \fmf{plain,tension=1}{a1',p1}

    % new boson lines
    \fmf{dashes,lab=$X$,tension=2}{b1,c1}

    % connection to external vertices had to be done first (before connecting
    % the X to YH
    \fmf{fermion}{o3,d1,o4}
    \fmf{fermion}{o1,d2,o2}
    \fmf{dashes,lab=$Y$,tension=1.2}{c1,d1}
    \fmf{dashes,lab=$H$,tension=1.2}{c1,d2}

    % add the filled circles after all the lines were established
    \fmfblob{.09w}{a1}
    \fmfblob{.09w}{a2}
    \fmfblob{.09w}{b1}

    % freeze the diagram before adding the 2nd and 3rd lines to the incoming
    % and outgoing protons
    \fmffreeze

    \fmfi{plain}{vpath (__i1,__a1) shifted (thick*(0,2))}
    \fmfi{plain}{vpath (__i1,__a1) shifted (thick*(0,-2))}
    \fmfi{plain}{vpath (__i2,__a2) shifted (thick*(0,2))}
    \fmfi{plain}{vpath (__i2,__a2) shifted (thick*(0,-2))}
    \fmfi{plain}{vpath (__a1',__p1) shifted (thick*(0,2))}
    \fmfi{plain}{vpath (__a1',__p1) shifted (thick*(0,-2))}
    \fmfi{plain}{vpath (__a2',__p2) shifted (thick*(0,2))}
    \fmfi{plain}{vpath (__a2',__p2) shifted (thick*(0,-2))}

\end{fmfgraph*}
\end{fmffile}

\end{document}


